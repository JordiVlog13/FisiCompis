\documentclass[a4paper,11pt]{article}
\usepackage[utf8]{inputenc}
\usepackage[T1]{fontenc}
\usepackage{lmodern}
\usepackage{hyperref}
\usepackage{geometry}
\geometry{margin=2.5cm}

\title{Simulación del Modelo de Ising}
\author{Tu Nombre}
\date{\today}

\begin{document}

\maketitle

\begin{abstract}
Este documento acompaña al notebook de Jupyter que implementa una simulación de Monte Carlo del modelo de Ising en una red bidimensional de tamaño \(N \times N\). Calcula métricas como energía, magnetización, capacidad calorífica, susceptibilidad y tiempo de correlación integrado.
\end{abstract}

\section*{Parámetros de la simulación}
\begin{itemize}
  \item \textbf{N}: tamaño de la red (número de espines por dimensión).
  \item \textbf{T}: temperatura del sistema.
  \item \textbf{J}: constante de interacción entre espines.
  \item \texttt{MC\_STEPS}: número de pasos Monte Carlo.
  \item \texttt{SAVE\_INTERVAL}: intervalo de guardado de snapshots.
  \item \texttt{SEED}: semilla para reproducibilidad.
\end{itemize}

\section*{Instalación de dependencias}
Para ejecutar el notebook en Jupyter, instala las librerías necesarias así:
\begin{verbatim}
%pip install numpy numba tqdm scipy matplotlib imageio
\end{verbatim}

\section*{Estructura del Notebook}
\begin{enumerate}
  \item \textbf{Celda 1:} Importaciones y parseo de argumentos.
  \item \textbf{Celda 2:} Parámetros, semilla y configuración de Numba.
  \item \textbf{Celda 3:} Funciones principales (\texttt{compute\_flip\_probs}, \texttt{mc\_sweep}, versiones vectorizadas).
  \item \textbf{Celda 4:} Simulación con \texttt{tqdm}, checkpointing y métricas.
  \item \textbf{Celda 5:} Post-procesado y guardado de datos.
  \item \textbf{Celda 6:} Plot de métricas.
  \item \textbf{Celda 7:} Animación a MP4 y GIF.
\end{enumerate}

\end{document}
